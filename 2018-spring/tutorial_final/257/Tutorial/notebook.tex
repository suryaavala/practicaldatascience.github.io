
% Default to the notebook output style

    


% Inherit from the specified cell style.




    
\documentclass[11pt]{article}

    
    
    \usepackage[T1]{fontenc}
    % Nicer default font (+ math font) than Computer Modern for most use cases
    \usepackage{mathpazo}

    % Basic figure setup, for now with no caption control since it's done
    % automatically by Pandoc (which extracts ![](path) syntax from Markdown).
    \usepackage{graphicx}
    % We will generate all images so they have a width \maxwidth. This means
    % that they will get their normal width if they fit onto the page, but
    % are scaled down if they would overflow the margins.
    \makeatletter
    \def\maxwidth{\ifdim\Gin@nat@width>\linewidth\linewidth
    \else\Gin@nat@width\fi}
    \makeatother
    \let\Oldincludegraphics\includegraphics
    % Set max figure width to be 80% of text width, for now hardcoded.
    \renewcommand{\includegraphics}[1]{\Oldincludegraphics[width=.8\maxwidth]{#1}}
    % Ensure that by default, figures have no caption (until we provide a
    % proper Figure object with a Caption API and a way to capture that
    % in the conversion process - todo).
    \usepackage{caption}
    \DeclareCaptionLabelFormat{nolabel}{}
    \captionsetup{labelformat=nolabel}

    \usepackage{adjustbox} % Used to constrain images to a maximum size 
    \usepackage{xcolor} % Allow colors to be defined
    \usepackage{enumerate} % Needed for markdown enumerations to work
    \usepackage{geometry} % Used to adjust the document margins
    \usepackage{amsmath} % Equations
    \usepackage{amssymb} % Equations
    \usepackage{textcomp} % defines textquotesingle
    % Hack from http://tex.stackexchange.com/a/47451/13684:
    \AtBeginDocument{%
        \def\PYZsq{\textquotesingle}% Upright quotes in Pygmentized code
    }
    \usepackage{upquote} % Upright quotes for verbatim code
    \usepackage{eurosym} % defines \euro
    \usepackage[mathletters]{ucs} % Extended unicode (utf-8) support
    \usepackage[utf8x]{inputenc} % Allow utf-8 characters in the tex document
    \usepackage{fancyvrb} % verbatim replacement that allows latex
    \usepackage{grffile} % extends the file name processing of package graphics 
                         % to support a larger range 
    % The hyperref package gives us a pdf with properly built
    % internal navigation ('pdf bookmarks' for the table of contents,
    % internal cross-reference links, web links for URLs, etc.)
    \usepackage{hyperref}
    \usepackage{longtable} % longtable support required by pandoc >1.10
    \usepackage{booktabs}  % table support for pandoc > 1.12.2
    \usepackage[inline]{enumitem} % IRkernel/repr support (it uses the enumerate* environment)
    \usepackage[normalem]{ulem} % ulem is needed to support strikethroughs (\sout)
                                % normalem makes italics be italics, not underlines
    

    
    
    % Colors for the hyperref package
    \definecolor{urlcolor}{rgb}{0,.145,.698}
    \definecolor{linkcolor}{rgb}{.71,0.21,0.01}
    \definecolor{citecolor}{rgb}{.12,.54,.11}

    % ANSI colors
    \definecolor{ansi-black}{HTML}{3E424D}
    \definecolor{ansi-black-intense}{HTML}{282C36}
    \definecolor{ansi-red}{HTML}{E75C58}
    \definecolor{ansi-red-intense}{HTML}{B22B31}
    \definecolor{ansi-green}{HTML}{00A250}
    \definecolor{ansi-green-intense}{HTML}{007427}
    \definecolor{ansi-yellow}{HTML}{DDB62B}
    \definecolor{ansi-yellow-intense}{HTML}{B27D12}
    \definecolor{ansi-blue}{HTML}{208FFB}
    \definecolor{ansi-blue-intense}{HTML}{0065CA}
    \definecolor{ansi-magenta}{HTML}{D160C4}
    \definecolor{ansi-magenta-intense}{HTML}{A03196}
    \definecolor{ansi-cyan}{HTML}{60C6C8}
    \definecolor{ansi-cyan-intense}{HTML}{258F8F}
    \definecolor{ansi-white}{HTML}{C5C1B4}
    \definecolor{ansi-white-intense}{HTML}{A1A6B2}

    % commands and environments needed by pandoc snippets
    % extracted from the output of `pandoc -s`
    \providecommand{\tightlist}{%
      \setlength{\itemsep}{0pt}\setlength{\parskip}{0pt}}
    \DefineVerbatimEnvironment{Highlighting}{Verbatim}{commandchars=\\\{\}}
    % Add ',fontsize=\small' for more characters per line
    \newenvironment{Shaded}{}{}
    \newcommand{\KeywordTok}[1]{\textcolor[rgb]{0.00,0.44,0.13}{\textbf{{#1}}}}
    \newcommand{\DataTypeTok}[1]{\textcolor[rgb]{0.56,0.13,0.00}{{#1}}}
    \newcommand{\DecValTok}[1]{\textcolor[rgb]{0.25,0.63,0.44}{{#1}}}
    \newcommand{\BaseNTok}[1]{\textcolor[rgb]{0.25,0.63,0.44}{{#1}}}
    \newcommand{\FloatTok}[1]{\textcolor[rgb]{0.25,0.63,0.44}{{#1}}}
    \newcommand{\CharTok}[1]{\textcolor[rgb]{0.25,0.44,0.63}{{#1}}}
    \newcommand{\StringTok}[1]{\textcolor[rgb]{0.25,0.44,0.63}{{#1}}}
    \newcommand{\CommentTok}[1]{\textcolor[rgb]{0.38,0.63,0.69}{\textit{{#1}}}}
    \newcommand{\OtherTok}[1]{\textcolor[rgb]{0.00,0.44,0.13}{{#1}}}
    \newcommand{\AlertTok}[1]{\textcolor[rgb]{1.00,0.00,0.00}{\textbf{{#1}}}}
    \newcommand{\FunctionTok}[1]{\textcolor[rgb]{0.02,0.16,0.49}{{#1}}}
    \newcommand{\RegionMarkerTok}[1]{{#1}}
    \newcommand{\ErrorTok}[1]{\textcolor[rgb]{1.00,0.00,0.00}{\textbf{{#1}}}}
    \newcommand{\NormalTok}[1]{{#1}}
    
    % Additional commands for more recent versions of Pandoc
    \newcommand{\ConstantTok}[1]{\textcolor[rgb]{0.53,0.00,0.00}{{#1}}}
    \newcommand{\SpecialCharTok}[1]{\textcolor[rgb]{0.25,0.44,0.63}{{#1}}}
    \newcommand{\VerbatimStringTok}[1]{\textcolor[rgb]{0.25,0.44,0.63}{{#1}}}
    \newcommand{\SpecialStringTok}[1]{\textcolor[rgb]{0.73,0.40,0.53}{{#1}}}
    \newcommand{\ImportTok}[1]{{#1}}
    \newcommand{\DocumentationTok}[1]{\textcolor[rgb]{0.73,0.13,0.13}{\textit{{#1}}}}
    \newcommand{\AnnotationTok}[1]{\textcolor[rgb]{0.38,0.63,0.69}{\textbf{\textit{{#1}}}}}
    \newcommand{\CommentVarTok}[1]{\textcolor[rgb]{0.38,0.63,0.69}{\textbf{\textit{{#1}}}}}
    \newcommand{\VariableTok}[1]{\textcolor[rgb]{0.10,0.09,0.49}{{#1}}}
    \newcommand{\ControlFlowTok}[1]{\textcolor[rgb]{0.00,0.44,0.13}{\textbf{{#1}}}}
    \newcommand{\OperatorTok}[1]{\textcolor[rgb]{0.40,0.40,0.40}{{#1}}}
    \newcommand{\BuiltInTok}[1]{{#1}}
    \newcommand{\ExtensionTok}[1]{{#1}}
    \newcommand{\PreprocessorTok}[1]{\textcolor[rgb]{0.74,0.48,0.00}{{#1}}}
    \newcommand{\AttributeTok}[1]{\textcolor[rgb]{0.49,0.56,0.16}{{#1}}}
    \newcommand{\InformationTok}[1]{\textcolor[rgb]{0.38,0.63,0.69}{\textbf{\textit{{#1}}}}}
    \newcommand{\WarningTok}[1]{\textcolor[rgb]{0.38,0.63,0.69}{\textbf{\textit{{#1}}}}}
    
    
    % Define a nice break command that doesn't care if a line doesn't already
    % exist.
    \def\br{\hspace*{\fill} \\* }
    % Math Jax compatability definitions
    \def\gt{>}
    \def\lt{<}
    % Document parameters
    \title{Tutorial}
    
    
    

    % Pygments definitions
    
\makeatletter
\def\PY@reset{\let\PY@it=\relax \let\PY@bf=\relax%
    \let\PY@ul=\relax \let\PY@tc=\relax%
    \let\PY@bc=\relax \let\PY@ff=\relax}
\def\PY@tok#1{\csname PY@tok@#1\endcsname}
\def\PY@toks#1+{\ifx\relax#1\empty\else%
    \PY@tok{#1}\expandafter\PY@toks\fi}
\def\PY@do#1{\PY@bc{\PY@tc{\PY@ul{%
    \PY@it{\PY@bf{\PY@ff{#1}}}}}}}
\def\PY#1#2{\PY@reset\PY@toks#1+\relax+\PY@do{#2}}

\expandafter\def\csname PY@tok@w\endcsname{\def\PY@tc##1{\textcolor[rgb]{0.73,0.73,0.73}{##1}}}
\expandafter\def\csname PY@tok@c\endcsname{\let\PY@it=\textit\def\PY@tc##1{\textcolor[rgb]{0.25,0.50,0.50}{##1}}}
\expandafter\def\csname PY@tok@cp\endcsname{\def\PY@tc##1{\textcolor[rgb]{0.74,0.48,0.00}{##1}}}
\expandafter\def\csname PY@tok@k\endcsname{\let\PY@bf=\textbf\def\PY@tc##1{\textcolor[rgb]{0.00,0.50,0.00}{##1}}}
\expandafter\def\csname PY@tok@kp\endcsname{\def\PY@tc##1{\textcolor[rgb]{0.00,0.50,0.00}{##1}}}
\expandafter\def\csname PY@tok@kt\endcsname{\def\PY@tc##1{\textcolor[rgb]{0.69,0.00,0.25}{##1}}}
\expandafter\def\csname PY@tok@o\endcsname{\def\PY@tc##1{\textcolor[rgb]{0.40,0.40,0.40}{##1}}}
\expandafter\def\csname PY@tok@ow\endcsname{\let\PY@bf=\textbf\def\PY@tc##1{\textcolor[rgb]{0.67,0.13,1.00}{##1}}}
\expandafter\def\csname PY@tok@nb\endcsname{\def\PY@tc##1{\textcolor[rgb]{0.00,0.50,0.00}{##1}}}
\expandafter\def\csname PY@tok@nf\endcsname{\def\PY@tc##1{\textcolor[rgb]{0.00,0.00,1.00}{##1}}}
\expandafter\def\csname PY@tok@nc\endcsname{\let\PY@bf=\textbf\def\PY@tc##1{\textcolor[rgb]{0.00,0.00,1.00}{##1}}}
\expandafter\def\csname PY@tok@nn\endcsname{\let\PY@bf=\textbf\def\PY@tc##1{\textcolor[rgb]{0.00,0.00,1.00}{##1}}}
\expandafter\def\csname PY@tok@ne\endcsname{\let\PY@bf=\textbf\def\PY@tc##1{\textcolor[rgb]{0.82,0.25,0.23}{##1}}}
\expandafter\def\csname PY@tok@nv\endcsname{\def\PY@tc##1{\textcolor[rgb]{0.10,0.09,0.49}{##1}}}
\expandafter\def\csname PY@tok@no\endcsname{\def\PY@tc##1{\textcolor[rgb]{0.53,0.00,0.00}{##1}}}
\expandafter\def\csname PY@tok@nl\endcsname{\def\PY@tc##1{\textcolor[rgb]{0.63,0.63,0.00}{##1}}}
\expandafter\def\csname PY@tok@ni\endcsname{\let\PY@bf=\textbf\def\PY@tc##1{\textcolor[rgb]{0.60,0.60,0.60}{##1}}}
\expandafter\def\csname PY@tok@na\endcsname{\def\PY@tc##1{\textcolor[rgb]{0.49,0.56,0.16}{##1}}}
\expandafter\def\csname PY@tok@nt\endcsname{\let\PY@bf=\textbf\def\PY@tc##1{\textcolor[rgb]{0.00,0.50,0.00}{##1}}}
\expandafter\def\csname PY@tok@nd\endcsname{\def\PY@tc##1{\textcolor[rgb]{0.67,0.13,1.00}{##1}}}
\expandafter\def\csname PY@tok@s\endcsname{\def\PY@tc##1{\textcolor[rgb]{0.73,0.13,0.13}{##1}}}
\expandafter\def\csname PY@tok@sd\endcsname{\let\PY@it=\textit\def\PY@tc##1{\textcolor[rgb]{0.73,0.13,0.13}{##1}}}
\expandafter\def\csname PY@tok@si\endcsname{\let\PY@bf=\textbf\def\PY@tc##1{\textcolor[rgb]{0.73,0.40,0.53}{##1}}}
\expandafter\def\csname PY@tok@se\endcsname{\let\PY@bf=\textbf\def\PY@tc##1{\textcolor[rgb]{0.73,0.40,0.13}{##1}}}
\expandafter\def\csname PY@tok@sr\endcsname{\def\PY@tc##1{\textcolor[rgb]{0.73,0.40,0.53}{##1}}}
\expandafter\def\csname PY@tok@ss\endcsname{\def\PY@tc##1{\textcolor[rgb]{0.10,0.09,0.49}{##1}}}
\expandafter\def\csname PY@tok@sx\endcsname{\def\PY@tc##1{\textcolor[rgb]{0.00,0.50,0.00}{##1}}}
\expandafter\def\csname PY@tok@m\endcsname{\def\PY@tc##1{\textcolor[rgb]{0.40,0.40,0.40}{##1}}}
\expandafter\def\csname PY@tok@gh\endcsname{\let\PY@bf=\textbf\def\PY@tc##1{\textcolor[rgb]{0.00,0.00,0.50}{##1}}}
\expandafter\def\csname PY@tok@gu\endcsname{\let\PY@bf=\textbf\def\PY@tc##1{\textcolor[rgb]{0.50,0.00,0.50}{##1}}}
\expandafter\def\csname PY@tok@gd\endcsname{\def\PY@tc##1{\textcolor[rgb]{0.63,0.00,0.00}{##1}}}
\expandafter\def\csname PY@tok@gi\endcsname{\def\PY@tc##1{\textcolor[rgb]{0.00,0.63,0.00}{##1}}}
\expandafter\def\csname PY@tok@gr\endcsname{\def\PY@tc##1{\textcolor[rgb]{1.00,0.00,0.00}{##1}}}
\expandafter\def\csname PY@tok@ge\endcsname{\let\PY@it=\textit}
\expandafter\def\csname PY@tok@gs\endcsname{\let\PY@bf=\textbf}
\expandafter\def\csname PY@tok@gp\endcsname{\let\PY@bf=\textbf\def\PY@tc##1{\textcolor[rgb]{0.00,0.00,0.50}{##1}}}
\expandafter\def\csname PY@tok@go\endcsname{\def\PY@tc##1{\textcolor[rgb]{0.53,0.53,0.53}{##1}}}
\expandafter\def\csname PY@tok@gt\endcsname{\def\PY@tc##1{\textcolor[rgb]{0.00,0.27,0.87}{##1}}}
\expandafter\def\csname PY@tok@err\endcsname{\def\PY@bc##1{\setlength{\fboxsep}{0pt}\fcolorbox[rgb]{1.00,0.00,0.00}{1,1,1}{\strut ##1}}}
\expandafter\def\csname PY@tok@kc\endcsname{\let\PY@bf=\textbf\def\PY@tc##1{\textcolor[rgb]{0.00,0.50,0.00}{##1}}}
\expandafter\def\csname PY@tok@kd\endcsname{\let\PY@bf=\textbf\def\PY@tc##1{\textcolor[rgb]{0.00,0.50,0.00}{##1}}}
\expandafter\def\csname PY@tok@kn\endcsname{\let\PY@bf=\textbf\def\PY@tc##1{\textcolor[rgb]{0.00,0.50,0.00}{##1}}}
\expandafter\def\csname PY@tok@kr\endcsname{\let\PY@bf=\textbf\def\PY@tc##1{\textcolor[rgb]{0.00,0.50,0.00}{##1}}}
\expandafter\def\csname PY@tok@bp\endcsname{\def\PY@tc##1{\textcolor[rgb]{0.00,0.50,0.00}{##1}}}
\expandafter\def\csname PY@tok@fm\endcsname{\def\PY@tc##1{\textcolor[rgb]{0.00,0.00,1.00}{##1}}}
\expandafter\def\csname PY@tok@vc\endcsname{\def\PY@tc##1{\textcolor[rgb]{0.10,0.09,0.49}{##1}}}
\expandafter\def\csname PY@tok@vg\endcsname{\def\PY@tc##1{\textcolor[rgb]{0.10,0.09,0.49}{##1}}}
\expandafter\def\csname PY@tok@vi\endcsname{\def\PY@tc##1{\textcolor[rgb]{0.10,0.09,0.49}{##1}}}
\expandafter\def\csname PY@tok@vm\endcsname{\def\PY@tc##1{\textcolor[rgb]{0.10,0.09,0.49}{##1}}}
\expandafter\def\csname PY@tok@sa\endcsname{\def\PY@tc##1{\textcolor[rgb]{0.73,0.13,0.13}{##1}}}
\expandafter\def\csname PY@tok@sb\endcsname{\def\PY@tc##1{\textcolor[rgb]{0.73,0.13,0.13}{##1}}}
\expandafter\def\csname PY@tok@sc\endcsname{\def\PY@tc##1{\textcolor[rgb]{0.73,0.13,0.13}{##1}}}
\expandafter\def\csname PY@tok@dl\endcsname{\def\PY@tc##1{\textcolor[rgb]{0.73,0.13,0.13}{##1}}}
\expandafter\def\csname PY@tok@s2\endcsname{\def\PY@tc##1{\textcolor[rgb]{0.73,0.13,0.13}{##1}}}
\expandafter\def\csname PY@tok@sh\endcsname{\def\PY@tc##1{\textcolor[rgb]{0.73,0.13,0.13}{##1}}}
\expandafter\def\csname PY@tok@s1\endcsname{\def\PY@tc##1{\textcolor[rgb]{0.73,0.13,0.13}{##1}}}
\expandafter\def\csname PY@tok@mb\endcsname{\def\PY@tc##1{\textcolor[rgb]{0.40,0.40,0.40}{##1}}}
\expandafter\def\csname PY@tok@mf\endcsname{\def\PY@tc##1{\textcolor[rgb]{0.40,0.40,0.40}{##1}}}
\expandafter\def\csname PY@tok@mh\endcsname{\def\PY@tc##1{\textcolor[rgb]{0.40,0.40,0.40}{##1}}}
\expandafter\def\csname PY@tok@mi\endcsname{\def\PY@tc##1{\textcolor[rgb]{0.40,0.40,0.40}{##1}}}
\expandafter\def\csname PY@tok@il\endcsname{\def\PY@tc##1{\textcolor[rgb]{0.40,0.40,0.40}{##1}}}
\expandafter\def\csname PY@tok@mo\endcsname{\def\PY@tc##1{\textcolor[rgb]{0.40,0.40,0.40}{##1}}}
\expandafter\def\csname PY@tok@ch\endcsname{\let\PY@it=\textit\def\PY@tc##1{\textcolor[rgb]{0.25,0.50,0.50}{##1}}}
\expandafter\def\csname PY@tok@cm\endcsname{\let\PY@it=\textit\def\PY@tc##1{\textcolor[rgb]{0.25,0.50,0.50}{##1}}}
\expandafter\def\csname PY@tok@cpf\endcsname{\let\PY@it=\textit\def\PY@tc##1{\textcolor[rgb]{0.25,0.50,0.50}{##1}}}
\expandafter\def\csname PY@tok@c1\endcsname{\let\PY@it=\textit\def\PY@tc##1{\textcolor[rgb]{0.25,0.50,0.50}{##1}}}
\expandafter\def\csname PY@tok@cs\endcsname{\let\PY@it=\textit\def\PY@tc##1{\textcolor[rgb]{0.25,0.50,0.50}{##1}}}

\def\PYZbs{\char`\\}
\def\PYZus{\char`\_}
\def\PYZob{\char`\{}
\def\PYZcb{\char`\}}
\def\PYZca{\char`\^}
\def\PYZam{\char`\&}
\def\PYZlt{\char`\<}
\def\PYZgt{\char`\>}
\def\PYZsh{\char`\#}
\def\PYZpc{\char`\%}
\def\PYZdl{\char`\$}
\def\PYZhy{\char`\-}
\def\PYZsq{\char`\'}
\def\PYZdq{\char`\"}
\def\PYZti{\char`\~}
% for compatibility with earlier versions
\def\PYZat{@}
\def\PYZlb{[}
\def\PYZrb{]}
\makeatother


    % Exact colors from NB
    \definecolor{incolor}{rgb}{0.0, 0.0, 0.5}
    \definecolor{outcolor}{rgb}{0.545, 0.0, 0.0}



    
    % Prevent overflowing lines due to hard-to-break entities
    \sloppy 
    % Setup hyperref package
    \hypersetup{
      breaklinks=true,  % so long urls are correctly broken across lines
      colorlinks=true,
      urlcolor=urlcolor,
      linkcolor=linkcolor,
      citecolor=citecolor,
      }
    % Slightly bigger margins than the latex defaults
    
    \geometry{verbose,tmargin=1in,bmargin=1in,lmargin=1in,rmargin=1in}
    
    

    \begin{document}
    
    
    \maketitle
    
    

    
    \section{Using Historical Collections as Data while using Python Imaging
Library (PIL) for image
processing}\label{using-historical-collections-as-data-while-using-python-imaging-library-pil-for-image-processing}

Before you can test and work with image processing through python, you
will need to have installed an additional library called PIL or Python
Imaging Library and a fork called \emph{Pillow} which has different
tools to manipulate the image once the library is installed as you will
see in the following sections.

\begin{quote}
The basic steps to install it are described
\href{https://pillow.readthedocs.io/en/latest/installation.html\#windows-installation}{here}
based on the type of user (MAC or WINDOWS); it also explains the which
version of pillow is supported by Python 3.X. Although, if you are using
anaconda, just make sure that your pillow version is up to date, since
PIL/pillow is within the packages installed in Anaconda.
\end{quote}

This is sample image in its original format and size: \textgreater{} 

    \begin{Verbatim}[commandchars=\\\{\}]
{\color{incolor}In [{\color{incolor}1}]:} \PY{c+c1}{\PYZsh{} All the imports needed for the entire notebook are listed on this section}
        \PY{k+kn}{from} \PY{n+nn}{PIL} \PY{k}{import} \PY{n}{Image}\PY{p}{,} \PY{n}{ImageFilter}
        \PY{k+kn}{from} \PY{n+nn}{matplotlib}\PY{n+nn}{.}\PY{n+nn}{pyplot} \PY{k}{import} \PY{n}{imshow} 
        
        \PY{k+kn}{import} \PY{n+nn}{os}  \PY{c+c1}{\PYZsh{} to work with a group of pictures in a folder}
\end{Verbatim}


    Once we have added all the necessary imports, we will work first on
importing the original image into the nothebook.

This section below displays the image as you saw it above with its
original size. if the image is within the same directory of the
notebook, we only need to add the name.extension of the image when
calling it; else we will need the full path of it when calling it.

In our case the image is saved within the the folder of this notebook.

    \begin{Verbatim}[commandchars=\\\{\}]
{\color{incolor}In [{\color{incolor}2}]:} \PY{c+c1}{\PYZsh{} we open the original file in JPEG extension}
        \PY{n}{myimage} \PY{o}{=} \PY{n}{Image}\PY{o}{.}\PY{n}{open}\PY{p}{(}\PY{l+s+s2}{\PYZdq{}}\PY{l+s+s2}{Rocky\PYZus{}Mountains.jpg}\PY{l+s+s2}{\PYZdq{}}\PY{p}{)}
        
        \PY{c+c1}{\PYZsh{} The original image  is displayed in windows photo viewer and }
        \PY{c+c1}{\PYZsh{} it comes up just as the one displayed at the beginning of the notebook.}
        \PY{c+c1}{\PYZsh{}myimage.show()}
        
        \PY{c+c1}{\PYZsh{} we modify the file by ssving it different extension}
        \PY{n}{myimage}\PY{o}{.}\PY{n}{save}\PY{p}{(}\PY{l+s+s2}{\PYZdq{}}\PY{l+s+s2}{Rocky\PYZus{}Mountains.png}\PY{l+s+s2}{\PYZdq{}}\PY{p}{)}
\end{Verbatim}


    Now lets change another feature of the image: its size. This is very
useful when storing or retrieving images from a server, so they will not
have any issues when retrieved.

Also, it is the way to make them thumbnails

    \begin{Verbatim}[commandchars=\\\{\}]
{\color{incolor}In [{\color{incolor}11}]:} \PY{c+c1}{\PYZsh{} specifies the size in pixels for the image}
         \PY{n}{tSize\PYZus{}300} \PY{o}{=} \PY{p}{(}\PY{l+m+mi}{300}\PY{p}{,}\PY{l+m+mi}{300}\PY{p}{)}
         \PY{c+c1}{\PYZsh{}tSize\PYZus{}700 = (700,700)}
         
         \PY{c+c1}{\PYZsh{}myimage2 = Image.open(myimage)}
         
         \PY{c+c1}{\PYZsh{} resizes the images }
         \PY{n}{myimage2} \PY{o}{=} \PY{n}{myimage}\PY{o}{.}\PY{n}{thumbnail}\PY{p}{(}\PY{n}{tSize\PYZus{}300}\PY{p}{)}
         \PY{c+c1}{\PYZsh{}image2 = myimage.thumbnail(tSize\PYZus{}700)}
         
         \PY{c+c1}{\PYZsh{} saves the images in their respective folders}
         \PY{c+c1}{\PYZsh{}image2.save(\PYZsq{}700/\PYZob{}\PYZcb{}\PYZus{}700\PYZob{}\PYZcb{}\PYZsq{}.format(myimage))}
         \PY{c+c1}{\PYZsh{}myimage2.save(\PYZsq{}300/\PYZob{}\PYZcb{}\PYZus{}300\PYZob{}\PYZcb{}\PYZsq{}.format(myimage))}
\end{Verbatim}


    \begin{Verbatim}[commandchars=\\\{\}]
<class 'NoneType'>

    \end{Verbatim}

    \begin{Verbatim}[commandchars=\\\{\}]
{\color{incolor}In [{\color{incolor} }]:} \PY{c+c1}{\PYZsh{}rotate the image}
        \PY{n}{myimage}\PY{o}{.}\PY{n}{rotate}\PY{p}{(}\PY{l+m+mi}{90}\PY{p}{)}\PY{o}{.}\PY{n}{save}\PY{p}{(}\PY{l+s+s2}{\PYZdq{}}\PY{l+s+s2}{Rocky\PYZus{}Mountains\PYZus{}90rotation.jpg}\PY{l+s+s2}{\PYZdq{}}\PY{p}{)}
\end{Verbatim}


    \begin{Verbatim}[commandchars=\\\{\}]
{\color{incolor}In [{\color{incolor} }]:} \PY{c+c1}{\PYZsh{} to change the color to black and white}
        \PY{n}{myimage}\PY{o}{.}\PY{n}{convert}\PY{p}{(}\PY{n}{mode} \PY{o}{=}\PY{l+s+s1}{\PYZsq{}}\PY{l+s+s1}{L}\PY{l+s+s1}{\PYZsq{}}\PY{p}{)}\PY{o}{.}\PY{n}{save}\PY{p}{(}\PY{l+s+s2}{\PYZdq{}}\PY{l+s+s2}{Rocky\PYZus{}Mountains\PYZus{}BW.jpg}\PY{l+s+s2}{\PYZdq{}}\PY{p}{)}
\end{Verbatim}


    \begin{Verbatim}[commandchars=\\\{\}]
{\color{incolor}In [{\color{incolor} }]:} \PY{c+c1}{\PYZsh{} To make a blurry image }
        \PY{c+c1}{\PYZsh{} The default value is set to 2 which does not make much of a difference }
        \PY{c+c1}{\PYZsh{} but if incremented you will see a notorious difference in the clarity of the image}
        \PY{n}{myimage}\PY{o}{.}\PY{n}{filter}\PY{p}{(}\PY{n}{ImageFilter}\PY{o}{.}\PY{n}{GaussianBlur}\PY{p}{(}\PY{l+m+mi}{15}\PY{p}{)}\PY{p}{)}\PY{o}{.}\PY{n}{save}\PY{p}{(}\PY{l+s+s2}{\PYZdq{}}\PY{l+s+s2}{Rocky\PYZus{}Mountains\PYZus{}Blurry.jpg}\PY{l+s+s2}{\PYZdq{}}\PY{p}{)}
\end{Verbatim}


    From this point forward we start to experiment with different features
of the image. The first one we will modify is the size of the image by
reducing it to a thumbnail, or simply changing the dimensions of the
image to make it fittable

    \begin{Verbatim}[commandchars=\\\{\}]
{\color{incolor}In [{\color{incolor} }]:} \PY{c+c1}{\PYZsh{}This block changes a group of .jpg images into .pngs}
        \PY{c+c1}{\PYZsh{}for i in os.listdir(\PYZsq{}.\PYZsq{}):}
        \PY{c+c1}{\PYZsh{}    if f.endswith(\PYZsq{}.jpg\PYZsq{}):}
        \PY{c+c1}{\PYZsh{}        i = image.open(f)}
        \PY{c+c1}{\PYZsh{}        fn,text = os.path.splitext(f)}
        \PY{c+c1}{\PYZsh{}        i.thumbnail(tSize)}
        \PY{c+c1}{\PYZsh{}        i.save(\PYZsq{}300/()\PYZus{}300()\PYZsq{}. format(fn, ftext))}
        \PY{c+c1}{\PYZsh{}        i.save(\PYZsq{}pngs/().png\PYZsq{}.format(fn))   \PYZsh{}\PYZlt{}\PYZhy{}\PYZhy{} this is to save them as pngs }
\end{Verbatim}


    \begin{Verbatim}[commandchars=\\\{\}]
{\color{incolor}In [{\color{incolor} }]:} \PY{c+c1}{\PYZsh{} imshow(array(myimage))}
        \PY{c+c1}{\PYZsh{} myimage.load()}
        \PY{c+c1}{\PYZsh{} http://www.pythonforbeginners.com/gui/how\PYZhy{}to\PYZhy{}use\PYZhy{}pillow}
        
        \PY{n}{I} \PY{n}{might} \PY{n}{need} \PY{n}{to} \PY{n}{elaborate}  \PY{o+ow}{in} \PY{n}{information} \PY{n}{about} \PY{n}{it} \PY{k+kn}{from} \PY{n+nn}{where} \PY{n}{the} \PY{n}{data} \PY{n}{comes} \PY{k+kn}{from} \PY{n+nn}{and} \PY{n}{how} \PY{n}{it} \PY{o+ow}{is} \PY{n}{used}
        
        \PY{n}{https}\PY{p}{:}\PY{o}{/}\PY{o}{/}\PY{n}{www}\PY{o}{.}\PY{n}{loc}\PY{o}{.}\PY{n}{gov}\PY{o}{/}\PY{n}{collections}
        
        \PY{n}{https}\PY{p}{:}\PY{o}{/}\PY{o}{/}\PY{n}{github}\PY{o}{.}\PY{n}{com}\PY{o}{/}\PY{n}{LibraryOfCongress}\PY{o}{/}\PY{n}{data}\PY{o}{\PYZhy{}}\PY{n}{exploration}\PY{o}{/}\PY{n}{blob}\PY{o}{/}\PY{n}{master}\PY{o}{/}\PY{n}{Accessing}\PY{o}{\PYZpc{}}\PY{k}{20images}\PYZpc{}20for\PYZpc{}20analysis.ipynb
        
        \PY{n}{https}\PY{p}{:}\PY{o}{/}\PY{o}{/}\PY{n}{github}\PY{o}{.}\PY{n}{com}\PY{o}{/}\PY{n}{LibraryOfCongress}\PY{o}{/}\PY{n}{data}\PY{o}{\PYZhy{}}\PY{n}{exploration}\PY{o}{/}\PY{n}{blob}\PY{o}{/}\PY{n}{master}\PY{o}{/}\PY{n}{ChronAm}\PY{o}{\PYZpc{}}\PY{k}{20API}\PYZpc{}20Samples.ipynb
        
        \PY{n}{https}\PY{p}{:}\PY{o}{/}\PY{o}{/}\PY{n}{github}\PY{o}{.}\PY{n}{com}\PY{o}{/}\PY{n}{LibraryOfCongress}\PY{o}{/}\PY{n}{data}\PY{o}{\PYZhy{}}\PY{n}{exploration}\PY{o}{/}\PY{n}{blob}\PY{o}{/}\PY{n}{master}\PY{o}{/}\PY{n}{OpenSearch}\PY{o}{.}\PY{n}{ipynb}
        
        \PY{n}{https}\PY{p}{:}\PY{o}{/}\PY{o}{/}\PY{n}{github}\PY{o}{.}\PY{n}{com}\PY{o}{/}\PY{n}{LibraryOfCongress}\PY{o}{/}\PY{n}{data}\PY{o}{\PYZhy{}}\PY{n}{exploration}\PY{o}{/}\PY{n}{blob}\PY{o}{/}\PY{n}{master}\PY{o}{/}\PY{n}{IIIF}\PY{o}{.}\PY{n}{ipynb}
        
        \PY{n}{http}\PY{p}{:}\PY{o}{/}\PY{o}{/}\PY{n}{www}\PY{o}{.}\PY{n}{scipy}\PY{o}{\PYZhy{}}\PY{n}{lectures}\PY{o}{.}\PY{n}{org}\PY{o}{/}\PY{n}{advanced}\PY{o}{/}\PY{n}{image\PYZus{}processing}\PY{o}{/}
        \PY{n}{http}\PY{p}{:}\PY{o}{/}\PY{o}{/}\PY{n}{opencv}\PY{o}{\PYZhy{}}\PY{n}{python}\PY{o}{\PYZhy{}}\PY{n}{tutroals}\PY{o}{.}\PY{n}{readthedocs}\PY{o}{.}\PY{n}{io}\PY{o}{/}\PY{n}{en}\PY{o}{/}\PY{n}{latest}\PY{o}{/}\PY{n}{py\PYZus{}tutorials}\PY{o}{/}\PY{n}{py\PYZus{}imgproc}\PY{o}{/}\PY{n}{py\PYZus{}table\PYZus{}of\PYZus{}contents\PYZus{}imgproc}\PY{o}{/}\PY{n}{py\PYZus{}table\PYZus{}of\PYZus{}contents\PYZus{}imgproc}\PY{o}{.}\PY{n}{html}
        
        \PY{n}{https}\PY{p}{:}\PY{o}{/}\PY{o}{/}\PY{n}{www}\PY{o}{.}\PY{n}{youtube}\PY{o}{.}\PY{n}{com}\PY{o}{/}\PY{n}{watch}\PY{err}{?}\PY{n}{v}\PY{o}{=}\PY{l+m+mf}{5E5}\PY{n}{mVVsrwZw}
        
        \PY{n}{https}\PY{p}{:}\PY{o}{/}\PY{o}{/}\PY{n}{www}\PY{o}{.}\PY{n}{youtube}\PY{o}{.}\PY{n}{com}\PY{o}{/}\PY{n}{watch}\PY{err}{?}\PY{n}{v}\PY{o}{=}\PY{l+m+mi}{6}\PY{n}{Qs3wObeWwc}
        
            
        \PY{c+c1}{\PYZsh{}f = misc.face()}
        \PY{c+c1}{\PYZsh{}misc.imsave(\PYZsq{}Rocky\PYZus{}Mountains.jpg\PYZsq{}, f) \PYZsh{} uses the Image module (PIL)}
        
        \PY{c+c1}{\PYZsh{}plt.imshow(f)}
        \PY{c+c1}{\PYZsh{}plt.show()}
\end{Verbatim}


    References: 1. http://www.pythonforbeginners.com/gui/how-to-use-pillow
2. https://www.youtube.com/watch?v=6Qs3wObeWwc 3.
https://github.com/adam-p/markdown-here/wiki/Markdown-Here-Cheatsheet\#blockquotes


    % Add a bibliography block to the postdoc
    
    
    
    \end{document}
